\documentclass[aps,prl,twocolumn]{revtex4-1}

\usepackage{graphicx}
\graphicspath{{figures/}}
\usepackage{amsmath}
\usepackage{SIunits}

\renewcommand{\vec}[1]{\mathbf{#1}}
\newcommand{\uvec}[1]{\hat{#1}}
\newcommand{\abs}[1]{\left\vert #1 \right\vert}
\newcommand{\real}[1]{\Re\left\{ #1 \right\}}
\newcommand{\imaginary}[1]{\Im\left\{ #1 \right\}}
\newcommand{\mytilde}{\raise.17ex\hbox{$\scriptstyle\mathtt{\sim}$}}
\newcommand{\pos}{\left(\mathbf{r} \right)}
\newcommand{\bigM}{\mathbf{M}_{mn}}
\newcommand{\bigN}{\mathbf{N}_{mn}}
%\newcommand{\pol}{\mathbf{\rh}}

\begin{document} 

\title{Understanding the vector spherical harmonics}

\author{David Ruffner}
\author{David G. Grier}
\affiliation{Department of Physics and Center for Soft Matter Research,
  New York University, New York, NY 10003}

\date{\today}
\begin{abstract}
  Working out the form of the vector spherical harmonics. 




\end{abstract}



\maketitle 

The vector spherical harmonics can be used to decompose a vector field 
in spherical polar coordinates. When the field is in this form we can 
more easily calculated the scattering off of spherical particles. In fact
each component of the incident field contributes to the scattered field
multiplied by the corresponding $a_n$ and $b_n$ of Lorentz-Mie theory
\cite{bohren_absorption_2008}. Each textbook uses it's own notation so 
it is hard to compare between the two. Here we'll go through the calculation
to compare Jackson \cite{jackson_classical_1998} to
 Gouesbet \cite{gouesbet_t-matrix_2010}. 

Jackson begins with the scalar wave equation,
\begin{equation}
  \label{eq:scalarwaveeq}
  \nabla^2 \psi - \frac{1}{c^2} \frac{\partial^2 \psi}{\partial t^2} = 0.
\end{equation}
Solutions can be written as an expansion in multipole terms,
\begin{equation}
  \label{eq:generalscalarsolution}
  \psi(\mathbf{r}) = \sum_{l,m} [A_{lm}z_l(kr)+B_{lm}z_l(kr)]Y_{lm}(\theta,\phi).
\end{equation}
We need to make a series of definitions.
$Y_{lm}$ are the spherical harmonics, defined by,
\begin{equation}
  \label{eq:sphericalharmonic}
  Y_{lm}(\theta,\phi) =  \tilde{P}_l^m(\cos \theta) e^{im\theta}
\end{equation}
Where $\tilde{P}_l^m(\cos \theta)$ are the normalized
associated Legendre polynomials (which are calculated by dbr\_plegendre.pro),
which are defined by,
\begin{equation}
  \label{eq:normalizedassociatedlegendrep}
  \tilde{P}_l^m(x) = N_p (-1)^m (1-x^2)^{m/2} \frac{d^m}{dx^m}P_l(x)
\end{equation}
Where $P_l(x)$ are the Legendre Polynomials defined by Rodrigues' formula
and the normalization is $N_p =\sqrt{\frac{(2l+1)(l-m)!}{4 \pi (l+m)!}}$.
And $z_l(kr)$ is one of the spherical Bessel functions, for instance
\begin{equation}
  \label{eq:sphericalbessel}
  j_l(x) = \sqrt{\frac{\pi}{2 x}}J_{l+1/2}(x), 
\end{equation}
Where $J_l(x)$ is a Bessel funtion of the first kind of order $l$. 

Next Jackson points out that Maxwell's equations can be written in the form
of the helmholtz equation and a divergence condition,
\begin{equation}
  \label{eq:maxwells}
  (\nabla^2 + k^2)\mathbf{E} = 0  \hspace{8mm}and\hspace{8mm}
  \nabla \cdot \mathbf{E} = 0
\end{equation}
with $\mathbf{H}$ given by
\begin{equation}
  \label{eq:maxwells2}
  \mathbf{H} = \frac{-i}{kZ_0} \nabla \times \mathbf{E}
\end{equation}
Where $Z_0=\frac{1}{c \epsilon_0}$ is the impedence of free space.
The hard part is enforcing the divergence condition. We can instead show
that $\mathbf{r} \cdot \mathbf{E}$ and $\mathbf{r} \cdot \mathbf{H}$ each
satisfy the scalar Helmoltz equation by using the vector identity,
\begin{equation}
  \label{eq:vectoridentity}
   \nabla^2 (\mathbf{r}\cdot \mathbf{A}) = \mathbf{r}\cdot \nabla^2\mathbf{A}
                                          +2 \nabla \cdot \mathbf{A}.
\end{equation}
Consequently we can expand $\mathbf{r} \cdot \mathbf{E}$ and
 $\mathbf{r} \cdot \mathbf{H}$ in terms of Eq.~(\ref{eq:generalscalarsolution})
which is why you need the two families of coefficients. It's interesting
that you only need two scalar fields to construct a vector field, but this
comes from the transverse nature of the fields.

To extract the electric an magnetic fields from the scalar fields we need
to use the angularmomentum operator, $\mathbf{L}$, defined by
\begin{equation}
  \label{eq:angularmomentum}
  \mathbf{L} = -i \mathbf{r} \times \nabla.
\end{equation}
Jackson shows that if,
\begin{equation}
  \label{eq:magneticmultipole}
  \mathbf{r}\cdot \mathbf{H}_{lm}^{(M)} = \frac{l(l+1)}{k} z_l(kr) 
                                        Y_{lm}(\theta,\phi)
\end{equation}
Then the electric field corresponding to this magnetic multipole of
order $l$,$m$ is,
\begin{equation}
  \label{eq:Efieldmagneticmultipole}
  \mathbf{E}_{lm}^{(M)} = Z_0 z_l(kr) \mathbf{L} Y_{lm}(\theta,\phi).
\end{equation}
Similarly for an electric multipole field,
\begin{equation}
  \label{eq:Efieldelectricmultipole}
  \mathbf{E}_{lm}^{(E)} =  \frac{i Z_0}{k} \nabla \times
                 ( z_l(kr) \mathbf{L} Y_{lm}(\theta,\phi)).
\end{equation}
These two sets of functions are the vector spherical harmonics which
are orthogonal funtions which allow us to write a general vector field
as a sum of them,
\begin{align}
  \label{eq:VSHsum}
  \mathbf{E} = \sum_{l,m} \left[\frac{i}{k} a_E(l,m) \nabla \times
                 ( z_l(kr) \mathbf{L} Y_{lm}(\theta,\phi)) \right. \\
                       \left.  +a_M(l,m) \mathbf{L} Y_{lm}(\theta,\phi)\right].
\end{align}
Here we depart slightly from Jackson who defines his vector spherical
harmonic function $\mathbf{X}_{lm}$ with a factor of $1/\sqrt{l(l+1)}$ which
we omit to match with other formulations. 

We can write these functions in spherical coordinates to enable comparison
with other works. Considering the magnetic term first,
\begin{align}
  \label{eq:magneticVSH}
  j_l(kr) \mathbf{L} Y_{lm}(\theta,\phi) &= 
                -i j_l(kr) \mathbf{r} \times \nabla Y_{lm}(\theta,\phi) \\
               =  
        -&\hat{\phi}i\frac{ \psi_l(kr)}{kr} \partial_\theta Y_{lm}(\theta,\phi)  \\
   - &\hat{\theta}\frac{ \psi_l(kr)}{kr} \frac{m Y_{lm}(\theta,\phi)}
                                         {\sin \theta},
\end{align}
Where $\psi_l(kr) = kr z_l(kr)$.
Next we evaluate the electric term,
\begin{align}
  \label{eq:electricVSH}
  \frac{-i}{k} \nabla \times j_l(kr) \mathbf{L} Y_{lm}(\theta,\phi) &=
        \frac{-i}{k} \nabla \times \mathbf{L}(j_l(kr) Y_{lm}(\theta,\phi)) \\   
       = &\hat{r} \frac{l(l+1)}{(kr)^2}\psi_l(kr)Y_{lm}(\theta,\phi) \\
         +&\hat{\theta} \frac{1}{kr} \psi_l^\prime(kr)
                       \partial_\theta Y_{lm}(\theta,\phi) \\
         +&\hat{\phi} \frac{i m}{kr} \psi_l^\prime(kr)
                       \frac{ Y_{lm}(\theta,\phi)}{\sin \theta}.
         %% +&\hat{\phi}
\end{align}
There are a number of steps in evaluating this term. We used the fact that
the angular momentum operator only operates on the angular variables
also we used (9.125) in Jackson and Eq.(8) in 
Gouesbet \cite{gouesbet_t-matrix_2010}. This gives the same result
as Eq.(1) and Eq.(2) in Gouesbet for the magnetic and the electric VSHs
respectively, except for a factor of $(-1)^m$.

For a circularly polarized plane wave travelling along the
z axis the coefficients are,
\begin{align}
  \label{eq:coefficients}
  a_M(l,m)_\pm &= i^l \sqrt{\frac{4 \pi (2 l+1)}{l (l+1)}} \delta_{m,\pm1} \\
  a_E(l,m)_\pm &= \pm i  a_M(l,m)_\pm
\end{align} 
These are different than the ones in Bohren and Huffman by the square root.

Numerically it's more stable to write the field in terms of the normalized
 angular functions $\tilde{\pi}_{mn}$ and $\tilde{\tau}_{mn}$, which
are defined by,
\begin{equation}
  \label{eq:pi_nm}
  \tilde{\pi}_{mn}(\cos \theta) =
              m\frac{ \tilde{P}_n^m (\cos \theta)}{\sin \theta}
\end{equation}
and
\begin{equation}
  \label{eq:pi_nm}
  \tilde{\tau}_{mn}(\cos \theta) = \partial_\theta \tilde{P}_n^m (\cos \theta).
\end{equation}
This gives us the vector spherical harmonics,
\begin{align}
  \label{eq:magneticVSH}
  j_l(kr) \mathbf{L} Y_{lm}(\theta,\phi) &= 
        -\hat{\phi}i\frac{ \psi_l(kr)}{kr} \tilde{\tau}_{mn}(\cos \theta) 
             e^{i m \phi}  \\
   - &\hat{\theta}\frac{ \psi_l(kr)}{kr}\tilde{\pi}_{mn}(\cos \theta)
             e^{i m \phi},
\end{align}
and,
\begin{align}
  \label{eq:electricVSH}
  \frac{-i}{k} \nabla \times j_l(kr) \mathbf{L} Y_{lm}(\theta,\phi) &=
         \hat{r} \frac{l(l+1)}{(kr)^2}\psi_l(kr)Y_{lm}(\theta,\phi) \\
         +&\hat{\theta} \frac{1}{kr} \psi_l^\prime(kr)
                      \tilde{\tau}_{mn}(\cos \theta) 
             e^{i m \phi}  \\
         +&\hat{\phi} \frac{i}{kr} \psi_l^\prime(kr)\tilde{\pi}_{mn}(\cos \theta)
             e^{i m \phi}.
         %% +&\hat{\phi}
\end{align}
%\bibliography{abbreviations,grier,tweezer,dgg}
%\input{soconveyor1.bbl}
\bibliography{tractorrangeLibrary}
%\bibliographystyle{publist}
\bibliographystyle{chicago}

\end{document} 
